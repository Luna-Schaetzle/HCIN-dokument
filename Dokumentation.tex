\documentclass[a4paper,12pt]{article}
\usepackage[utf8]{inputenc}
\usepackage[T1]{fontenc}
\usepackage{lmodern}
\usepackage{geometry}
\usepackage{graphicx}
\usepackage{setspace}
\usepackage{hyperref}
\usepackage[backend=bibtex,style=alphabetic]{biblatex}
\geometry{margin=1in}
\doublespacing
\addbibresource{references.bib}

\begin{document}

% Title Page
\begin{titlepage}
    \centering
    \vspace*{1.5cm}
    \Huge\textbf{Mensch-KI-Interaktion: Chancen und Herausforderungen}

    \vspace{1cm}
    \Large\textbf{Begleitdokumentation zur Präsentation}

    \vspace{2cm}
    \textbf{Präsentiert von:}

    \vspace{0.5cm}
    \large
    Luna \& Florian

    \vfill
    \large \today
\end{titlepage}

% Table of Contents
\tableofcontents
\newpage

% Einleitung
\section{Einleitung: Einführung in die Mensch-KI-Interaktion}
Die Mensch-KI-Interaktion beschreibt die Zusammenarbeit zwischen Menschen und künstlichen Intelligenzsystemen. Sie umfasst verschiedene Formen der Kommunikation, wie Sprache, Bild- und Textverarbeitung, sowie emotionale und soziale Aspekte. Ziel ist es, eine effektive, natürliche Interaktion zu ermöglichen, die den Menschen in verschiedenen Lebens- und Arbeitsbereichen unterstützt.

% Interaktionsformen
\section{Interaktionsformen zwischen Mensch und KI}
\textbf{Präsentiert von: Luna}

Die Interaktion zwischen Mensch und KI lässt sich in verschiedene Formen einteilen, darunter:
\begin{itemize}
    \item \textbf{Sprach- und Hörverarbeitung:} Nutzung von Sprachassistenten und Spracherkennungssystemen.
    \item \textbf{Visuelle Verarbeitung:} Bilderkennung und Klassifikation.
    \item \textbf{Haptische und schriftliche Verarbeitung:} Textgenerierung und Simulation menschlicher Kommunikation.
\end{itemize}

% Grundlagen der Bilderkennung
\section{Grundlagen der Bilderkennung}
\textbf{Präsentiert von: Florian}

Bilderkennung ist eine Schlüsseltechnologie der künstlichen Intelligenz, die darauf abzielt, visuelle Daten zu analysieren und zu interpretieren. Grundlagen umfassen:
\begin{itemize}
    \item Pixelbasierte Verarbeitung von Bilddaten.
    \item Nutzung von Algorithmen wie Convolutional Neural Networks (CNNs).
    \item Anwendungen in Objekterkennung, Gesichtserkennung und medizinischer Bildanalyse.
\end{itemize}

% Technische Analyse der Bilderkennung
\section{Technische Analyse der Bilderkennung}
\textbf{Präsentiert von: Florian}

Eine technische Analyse umfasst die Verarbeitungsschritte, Datenanforderungen und Algorithmen. CNNs bestehen aus Schichten, die Features wie Kanten, Formen und Texturen extrahieren. Herausforderungen sind:
\begin{itemize}
    \item Rechenleistung.
    \item Trainingsdaten.
    \item Bias in den Daten.
\end{itemize}

% Aktuelle Schnittstellen und Anwendungen der Bilderkennung
\section{Aktuelle Schnittstellen und Anwendungen der Bilderkennung}
\textbf{Präsentiert von: Florian}

\textbf{Anwendungsbereiche:}
\begin{itemize}
    \item Sicherheit: Gesichtserkennung.
    \item Medizin: Diagnostik.
    \item Unterhaltung: Bildfilter und Effekte.
\end{itemize}

% Datenschutz bei KI
\section{Datenschutz bei KI}
\textbf{Präsentiert von: Florian \& Luna}

Der Datenschutz ist eine zentrale Herausforderung in der KI. Themen umfassen:
\begin{itemize}
    \item Speicherung sensibler Daten.
    \item Gesetzliche Anforderungen wie die DSGVO.
    \item Schutz vor Missbrauch und unbefugtem Zugriff.
\end{itemize}

% KI in der Arbeitswelt
\section{KI in der Arbeitswelt: Anwendungen in verschiedenen Branchen}
\textbf{Präsentiert von: Luna}

Künstliche Intelligenz verändert die Arbeitswelt erheblich. Anwendungen finden sich in:
\begin{itemize}
    \item Industrie 4.0: Automatisierung.
    \item Gesundheitswesen: Diagnostik.
    \item Logistik: Routenplanung und Lieferkettenoptimierung.
\end{itemize}

% Vermenschlichung der KI
\section{Vermenschlichung der KI und Uncanny Valley-Effekt}
\textbf{Präsentiert von: Luna}

Die Vermenschlichung der KI umfasst die Simulation menschlicher Merkmale. Der Uncanny Valley-Effekt beschreibt die Befremdlichkeit, wenn KI zu menschenähnlich wird.

% Emotionale Themen
\section{KI und emotionale Themen: Liebe, Trauer, Einsamkeit}
\textbf{Präsentiert von: Luna}

KI wird zunehmend in emotionalen Kontexten eingesetzt. Beispiele:
\begin{itemize}
    \item Chatbots als emotionale Begleiter.
    \item KI-gestützte Therapieansätze.
    \item Risiken: Abhängigkeit und fehlender menschlicher Kontakt.
\end{itemize}

% Psychologische Auswirkungen
\section{Psychologische Auswirkungen von KI}
\textbf{Präsentiert von: Flo \& Luna}

KI beeinflusst das menschliche Verhalten, von verstärktem Komfort bis hin zu Herausforderungen wie Arbeitsplatzverlust und Abhängigkeit.

% Gesellschaftliche Implikationen
\section{Gesellschaftliche Implikationen}
\textbf{Präsentiert von: Florian}


Die Integration von KI in die Gesellschaft wirft Fragen zu Ethik, Arbeitsmarkt und sozialer Gerechtigkeit auf.

% Praktische Erfahrungen
\section{Praktische Erfahrungen mit KI}
\textbf{Präsentiert von: Luna \& Flo}

Im Rahmen der Arbeit wurden verschiedene KI-Systeme getestet und analysiert, um praktische Erkenntnisse zu gewinnen.

% Schlussfolgerung
\section{Schlussfolgerung und Ausblick}
\textbf{Präsentiert von: Luna \& Florian}

Künstliche Intelligenz bietet enormes Potenzial, erfordert jedoch auch sorgfältige Analyse und ethische Abwägungen. Der Ausblick zeigt, dass KI weiter an Bedeutung gewinnen wird.

% Literaturverzeichnis
\newpage
\printbibliography

\end{document}
